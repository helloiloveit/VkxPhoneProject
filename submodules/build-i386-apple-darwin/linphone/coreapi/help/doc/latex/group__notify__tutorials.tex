\section{Generic subscribe/notify example}
\label{group__notify__tutorials}\index{Generic subscribe/notify example@{Generic subscribe/notify example}}
$\ast$\-This program is a {\itshape very} simple usage example of liblinphone. $\ast$\-It demonstrates how to listen to a S\-I\-P subscription. $\ast$\-It then sends notify requests back periodically. $\ast$first argument must be like sip\-:{\tt jehan@sip.\-linphone.\-org} , second must be password . \par
 $\ast$ex registration sip\-:{\tt jehan@sip.\-linphone.\-org} secret \par
Registration is cleared on S\-I\-G\-I\-N\-T \par
 
\begin{DoxyCodeInclude}

\textcolor{comment}{/*}
\textcolor{comment}{linphone}
\textcolor{comment}{Copyright (C) 2010  Belledonne Communications SARL }
\textcolor{comment}{}
\textcolor{comment}{This program is free software; you can redistribute it and/or}
\textcolor{comment}{modify it under the terms of the GNU General Public License}
\textcolor{comment}{as published by the Free Software Foundation; either version 2}
\textcolor{comment}{of the License, or (at your option) any later version.}
\textcolor{comment}{}
\textcolor{comment}{This program is distributed in the hope that it will be useful,}
\textcolor{comment}{but WITHOUT ANY WARRANTY; without even the implied warranty of}
\textcolor{comment}{MERCHANTABILITY or FITNESS FOR A PARTICULAR PURPOSE.  See the}
\textcolor{comment}{GNU General Public License for more details.}
\textcolor{comment}{}
\textcolor{comment}{You should have received a copy of the GNU General Public License}
\textcolor{comment}{along with this program; if not, write to the Free Software}
\textcolor{comment}{Foundation, Inc., 59 Temple Place - Suite 330, Boston, MA  02111-1307, USA.}
\textcolor{comment}{*/}

\textcolor{preprocessor}{#ifdef IN\_LINPHONE}
\textcolor{preprocessor}{}\textcolor{preprocessor}{#include "linphonecore.h"}
\textcolor{preprocessor}{#else}
\textcolor{preprocessor}{}\textcolor{preprocessor}{#include "linphone/linphonecore.h"}
\textcolor{preprocessor}{#endif}
\textcolor{preprocessor}{}
\textcolor{preprocessor}{#include <signal.h>}

\textcolor{keyword}{static} bool\_t running=TRUE;

\textcolor{keyword}{static} \textcolor{keywordtype}{void} stop(\textcolor{keywordtype}{int} signum)\{
        running=FALSE;
\}

\textcolor{keyword}{static} \textcolor{keywordtype}{void} registration\_state\_changed(\textcolor{keyword}{struct} \_LinphoneCore *lc, 
      LinphoneProxyConfig *cfg, LinphoneRegistrationState cstate, \textcolor{keyword}{const} \textcolor{keywordtype}{char} *message)\{
                printf(\textcolor{stringliteral}{"New registration state %s for user id [%s] at proxy [%s]\(\backslash\)n"}
                                ,linphone_registration_state_to_string(cstate)
                                ,linphone_proxy_config_get_identity(cfg)
                                ,linphone_proxy_config_get_addr(cfg));
\}

LinphoneCore *lc;
\textcolor{keywordtype}{int} main(\textcolor{keywordtype}{int} argc, \textcolor{keywordtype}{char} *argv[])\{
        LinphoneCoreVTable vtable=\{0\};

        \textcolor{keywordtype}{char}* identity=NULL;
        \textcolor{keywordtype}{char}* password=NULL;

        \textcolor{comment}{/* takes   sip uri  identity from the command line arguments */}
        \textcolor{keywordflow}{if} (argc>1)\{
                identity=argv[1];
        \}

        \textcolor{comment}{/* takes   password from the command line arguments */}
        \textcolor{keywordflow}{if} (argc>2)\{
                password=argv[2];
        \}

        signal(SIGINT,stop);

\textcolor{preprocessor}{#ifdef DEBUG}
\textcolor{preprocessor}{}        linphone_core_enable_logs(NULL); \textcolor{comment}{/*enable liblinphone logs.*/}
\textcolor{preprocessor}{#endif}
\textcolor{preprocessor}{}        \textcolor{comment}{/* }
\textcolor{comment}{         Fill the LinphoneCoreVTable with application callbacks.}
\textcolor{comment}{         All are optional. Here we only use the registration\_state\_changed callbacks}
\textcolor{comment}{         in order to get notifications about the progress of the registration.}
\textcolor{comment}{         */}
        vtable.registration_state_changed=registration\_state\_changed;

        \textcolor{comment}{/*}
\textcolor{comment}{         Instanciate a LinphoneCore object given the LinphoneCoreVTable}
\textcolor{comment}{        */}
        lc=linphone_core_new(&vtable,NULL,NULL,NULL);

        LinphoneProxyConfig* proxy\_cfg;
        \textcolor{comment}{/*create proxy config*/}
        proxy\_cfg = linphone_proxy_config_new();
        \textcolor{comment}{/*parse identity*/}
        LinphoneAddress *from = linphone_address_new(identity);
        \textcolor{keywordflow}{if} (from==NULL)\{
                printf(\textcolor{stringliteral}{"%s not a valid sip uri, must be like sip:toto@sip.linphone.org \(\backslash\)n"},identity);
                \textcolor{keywordflow}{goto} end;
        \}
                LinphoneAuthInfo *info;
                \textcolor{keywordflow}{if} (password!=NULL)\{
                        info=linphone_auth_info_new(
      linphone_address_get_username(from),NULL,password,NULL,NULL); \textcolor{comment}{/*create authentication structure from
       identity*/}
                        linphone_core_add_auth_info(lc,info); \textcolor{comment}{/*add authentication info to LinphoneCore*/}
                \}

                \textcolor{comment}{// configure proxy entries}
                linphone_proxy_config_set_identity(proxy\_cfg,identity); \textcolor{comment}{/*set identity with user name and
       domain*/}
                \textcolor{keyword}{const} \textcolor{keywordtype}{char}* server\_addr = linphone_address_get_domain(from); \textcolor{comment}{/*extract domain address from
       identity*/}
                linphone_proxy_config_set_server_addr(proxy\_cfg,server\_addr); \textcolor{comment}{/* we assume domain = proxy
       server address*/}
                linphone_proxy_config_enable_register(proxy\_cfg,TRUE); \textcolor{comment}{/*activate registration for this
       proxy config*/}
                linphone_address_destroy(from); \textcolor{comment}{/*release resource*/}

                linphone_core_add_proxy_config(lc,proxy\_cfg); \textcolor{comment}{/*add proxy config to linphone core*/}
                linphone_core_set_default_proxy(lc,proxy\_cfg); \textcolor{comment}{/*set to default proxy*/}


        \textcolor{comment}{/* main loop for receiving notifications and doing background linphonecore work: */}
        \textcolor{keywordflow}{while}(running)\{
                linphone_core_iterate(lc); \textcolor{comment}{/* first iterate initiates registration */}
                ms\_usleep(50000);
        \}

        linphone_core_get_default_proxy(lc,&proxy\_cfg); \textcolor{comment}{/* get default proxy config*/}
        linphone_proxy_config_edit(proxy\_cfg); \textcolor{comment}{/*start editing proxy configuration*/}
        linphone_proxy_config_enable_register(proxy\_cfg,FALSE); \textcolor{comment}{/*de-activate registration for this proxy
       config*/}
        linphone_proxy_config_done(proxy\_cfg); \textcolor{comment}{/*initiate REGISTER with expire = 0*/}

        \textcolor{keywordflow}{while}(linphone\_proxy\_config\_get\_state(proxy\_cfg) !=  
      LinphoneRegistrationCleared)\{
                linphone_core_iterate(lc); \textcolor{comment}{/*to make sure we receive call backs before shutting down*/}
                ms\_usleep(50000);
        \}

end:
        printf(\textcolor{stringliteral}{"Shutting down...\(\backslash\)n"});
        linphone_core_destroy(lc);
        printf(\textcolor{stringliteral}{"Exited\(\backslash\)n"});
        \textcolor{keywordflow}{return} 0;
\}

\end{DoxyCodeInclude}
 